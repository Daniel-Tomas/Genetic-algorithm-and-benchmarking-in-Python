\documentclass[11pt, a4paper, titlepage]{article}
\setlength{\parindent}{0pt}

%%%%%%%% paquetes %%%%%%%%
%\usepackage[lmargin=2cm,rmargin=2cm,top=1.5cm,bottom=2cm]{geometry}
\usepackage[T1]{fontenc}
\usepackage[utf8]{inputenc}
\usepackage[spanish,es-tabla]{babel}
\usepackage{amsmath}
\usepackage{amssymb,amsfonts,latexsym,cancel}
\usepackage{fancyhdr}
\usepackage{titlesec}
\usepackage{titling}
\usepackage{anyfontsize}
\usepackage{color}

\usepackage{csquotes}   
\usepackage[style=numeric-comp, sorting=none, block=par]{biblatex} 
\DeclareFieldFormat{title}{\bfseries\emph{#1}}
\definecolor{softblack}{RGB}{74, 71, 71} 
\DeclareFieldFormat{howpublished}{\textcolor{softblack}{\mdseries{#1}}}
\DeclareFieldFormat{labelnumberwidth}{\mkbibbold{#1\adddot}}
\setlength\bibitemsep{2.5\itemsep}
\renewcommand*{\newunitpunct}{\addspace}
\renewcommand*{\finentrypunct}{\addspace}

%\usepackage[colorlinks = true,
			%linkcolor = blue,
			%urlcolor = black,
			%citecolor = blue ]{hyperref}

%%%%%%%% encabezado y pie de página %%%%%%%%
\pagestyle{fancy}
\fancyhead{}
\fancyfoot{}
\fancyfoot[R]{\thepage}
\fancyfoot[L]{2º Trabajo Optimización Heurística}
\renewcommand{\headrulewidth}{0pt}

%%%%%%%% formato de títulos y subtítulos %%%%%%%%

\definecolor{gray75}{gray}{0.75}
\newcommand{\hsp}{\hspace{20pt}}
\titleformat{\section}[hang]{\huge\bfseries}{\thesection\hsp\textcolor{gray75}{|}\hsp}{0pt}{\huge\bfseries}
\titlespacing{\section}{0pt}{0pt}{15pt}
\titleformat{\subsection}[hang]{\Large\bfseries}{\thesubsection\hsp}{0pt}{\Large\bfseries}
\titlespacing{\subsection}{0pt}{35pt}{15pt}
\titleformat{\subsubsection}[hang]{\large\bfseries}{\thesubsubsection\hspace{10pt}}{0pt}{\large\bfseries} 
\titlespacing{\subsubsection}{0pt}{20pt}{0pt}

%\titleformat{\section}[block]{\LARGE\bfseries}{\thesection.}{1mm}{}
%\titlespacing{\section}{0pc}{5.5ex}{1pc}
%\titleformat{\subsection}[block]{\Large\bfseries}{\thesubsection.}{1mm}{}
%\titlespacing{\subsection}{1.5pc}{5.5ex}{1pc}



\begin{document}

	\begin{titlepage}
    	\begin{center}
        	\hrulefill

        	\vspace{0.5cm}
        	{\bf\fontsize{25}{0}{\selectfont{Optimización Heurística\\[0.5cm]}}}
        	\fontsize{15}{0}{\selectfont{Optimizar una calificación\\[0.5cm]}}
        	\hrulefill
        	\vspace{6.0cm}
    	\end{center}

    	\centering
    	{\Large Daniel Tomás Sánchez\\ Aarón Cabero Blanco \\ Pablo Bautista 				Frías \par}
    	\vspace{2cm}
    	{\Large 03/01/2020 \par}
	\end{titlepage}

\newpage

%\renewcommand{\contentsname}{\fontsize{22}{0}{\selectfont{Índice}}}
%{\Large \tableofcontents}

\tableofcontents

\newpage

\section{Introducción}
Durante la realización de esta práctica se ha planteado un problema de optimización para su posterior resolución. Se explicará en detalle el problema así como las operaciones realizadas para obtener la solución óptima con el algoritmo. Posteriormente se resolverá el problema con diferentes datos de entrada para así poder ver los distintos resultados obtenidos.

\newpage

\section{Descripción}
El problema planteado consiste en la optimización de la nota media obtenida en un conjunto de asignaturas por un alumno. Para que el caso sea lo más real posible, se han añadido distintas variables que afectan al resultado óptimo del problema, como por ejemplo, las horas de las que dispone el alumno para estudiar o la dificultad que tiene cada unas de las asignaturas.

\newpage

\section{Función Fitness}
Aquí exponemos y explicamos nuestra función Fitness

\section{Ejemplos}
Aquí irán distintos ejemplos del problema

\subsection{Resultados con 1 repetición}
Aquí mostraremos ejemplos con una repetición

\subsection{Resultados con varias repeticiones}
Aquí mostraremos ejemplos con varias repeticiones y sus gráficos



\end{document}